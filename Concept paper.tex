\documentclass[12]{article}

\begin{document}
\title{The Relevance of Mobile Phones And Their Effects To Uganda As A Nation}
\author{Ndagga Nicholas Brenden 15/u/11200/eve}
\date{May, 21, 2017}
\maketitle

\newpage


\subsection{Abstract}
It is common cause that the advent of the mobile telecommunication, particularly the mobile phone has been immensely beneficial to the developing countries like Uganda. Not only has facilitated and improved communication between individuals, but has also enabled economies to grow faster. This paper explores an additional benefit that derives from having access to the mobile phones.it explores the development uses of the mobile phones in Uganda.it examines the relationship between the economic 

\section{Introduction}
Unlike most other end products, the end product in the mobile telecommunication industry, namely the mobile phone is used extensively  Uganda by people that have access to the mobile phone to upgrade themselves economically  and socially in a remarkably wide variety ways. These  ways range from stating up enterprise in which the mobile phone play a crucial role, to utilizing the mobile phone applications and the platforms constructed by the mobile phone network operators(MNOS).These economic and social upgrading initiatives are mostly developed in nature that they raise the human and productive capacity of the economies in which they are embedded.
The paper examines the developmental use of mobile phones in Uganda.it commences with the expansion of MTN mobile money services which consist of three types M-transfer-payments and  M-financials services which enable social upgrading to take place-payments also lead to social downgrading

\section{Background}
	It is common cause that the advent of the mobile telecommunication, particularly the mobile phone has been immensely beneficial to the developing countries like Uganda. Not only has facilitated and improved communication between individuals, but has also enabled economies to grow faster. This paper explores an additional benefit that derives from having access  to the mobile phones.it explores the development uses of the mobile phones in Uganda.it examines the relationship between the economic upgrading and the social upgrading or downgrading that result fro the developmental uses of the mobile phones.it is done by means of case studies. For example it explores the use of MTN mobile money facility focusing on the developmental benefit or the downgrade resulting from its use.

\section{Statement of the Problem}
The project focus on the social, economic upgrading relevance of mobile phones in Uganda and the social downgrading results from the use of mobile phones. The evolution of the various telecommunication companies and the various services offered to improve the communication between people.

\section{Objectives}

\subsection{Main Objective}
To look at the significances mobile phones the various social and economic developmental benefits in Uganda.to identify the mechanisms through which the mobile phone can provide economic benefit to both the consumers and the producers in Uganda first they can improve access to and use of information thereby reducing search costs improving coordination among agents and increasing market efficiency.

\section{Literature review}
The literature review therefore focuses on the use of mobile phones in Uganda in the different parts of the country and how mobile phone usage is perceived by most Ugandans.it is acknowledge that it is difficult to draw conclusive arguments of mobile phone use as culture, values and belief systems differ around the world play a part in the perception and use of technology.

\section{Methodology}
The study was descriptive in nature, with the use of the survey to gather the data. The research method selected for this study was a survey approach because it allowed for a larger sample to be gathered, as opposed to interviews or other forms of data gathering. The survey was used to obtain the behavior, opinion and attitudes of many Ugandans with regard to their use of the mobile phones.
The elevated societal status of early adopters of the mobile phones has led to the proud display of the of the handsets owners


\subsection{Conclusion}
This study set out to explore the use of mobile phones in uganda as a contribution to a variety of related fields. Media research has largely ignored the phenomennon of the mobile phones whereas it has increasingly led to more access to the web and other services on the internet among the youth.The availability of other mobile features such as bluetooth or an internal camera define the handsets predefined support for a rannge of other application.Mobile phones can be used to both provide moe interactive means of telling the news from various angles but also to enlarge the potential of various readers and viewer.





\end{document}